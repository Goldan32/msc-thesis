\chapter{Improvements to the Basic Project}

In the previous chapter the focus was on creating a project that can target a dual-core STM32 microcontroller. It was however only a basic "Hello World" type project that can be massively improved upon. In this chapter I will demonstrate the steps that were taken before attempting to build a proper example project.

\section{UART}

Establishing a serial connection between the target MCU and the host PC is usually the most important first step at the start of a new embedded project. While an UART line may not have full debugging capabilities, it allows for quick and relatively easy text based communication with the host. These traces are a very effective tool in determining where a program gets stuck in a loop or panics.

The following code brings up the UART interface that is linked to the ST-Link connector on the board. In this project, this UART interface is used for debugging purposes, so only the rx line is not in use. Rust generates warnings for variables that are not used, this can be avoided by prepending an underscore to the unused variable. The serial line is configured to operate with 19200 baudrate and without flow control.

\begin{lstlisting}[language=C,frame=single,float=!ht]
    let tx = gpiod.pd8.into_alternate();
    let _rx = gpiod.pd9.into_alternate();
    let serial = dp
        .USART3
        .serial((tx, _rx), 19200.bps(), ccdr.peripheral.USART3, &ccdr.clocks)
        .unwrap();
    let (mut tx, _rx) = serial.split();
\end{lstlisting}

\section{Debugger}

Using traces on a serial line for detecting errors in the code may not be sufficient in all cases. Moreover the overhead of printing to this peripheral can disrupt the timing of certain parts of the program. In these cases configuring a debugger becomes a necessity.

All STM32 development boards are equipped with an on-board ST-Link debugger. ST-Link connects to the host PC through a USB interface. The ST-Link is able to facilitate debugging through different modes of communication for example single-wire interface module (SWIM), serial wire debugging (SWD) and Joint Test Action Group (JTAG) of which the latter is used most often.

The official IDE provided by STM, STMCubeIDE includes full debugging capabilities, however these tools can only handle C and C++ code. The Rust ecosystem does not currently have a standard tool nor does STM provide debugging tools for other languages. Most often though Rust developers will use openocd, gdb and VSCode.

\subsection{OpenOCD}

Open On-Chip Debugger (OpenOCD) is an open source tool that provides debugging and in-system programming capabilities for embedded devices such as this STM32 microcontroller. It serves as a bridge between the development environment on a host machine and the microcontroller's hardware, facilitating the debugging process.

OpenOCD supports various hardware interfaces, such as JTAG, SWD, and various proprietary interfaces provided by different microcontroller vendors. These interfaces are crucial for establishing a connection between the host machine and the microcontroller, enabling the exchange of debugging information. The software can be configured to the parameters of the target device, such as the CPU architecture, target voltage, and other specific settings. This ensures that the debugger communicates effectively with the microcontroller.

OpenOCD acts as a GDB (GNU Debugger) server, providing a standardized interface for debugging tools. GDB is a full fetched debugger but only its server part is used in this configuration. OpenOCD enables GDB to connect to the target microcontroller, allowing developers to interact with and debug their code. The program also supports in-system programming, allowing users to flash the firmware onto the memory of the microcontroller. This is essential for updating or loading new firmware onto the device during the development and debugging process, however it is also possible to start debugging without flashing in new software, which is ideal for our case as the flashing process for this project is non-trivial due to the two cores.

OpenOCD can be integrated with various Integrated Development Environments and toolchains, providing a seamless debugging experience for developers using different development environments. Being an open-source project, it benefits from a vibrant community that contributes to its development and supports a wide range of hardware platforms. It also allows users to customize and extend its functionality based on their specific debugging requirements.

In the case of this project, the OpenOCD configuration is already provided for the evaluation board. \cite{OpenocdConfigFile}

\subsection{GDB}

The GNU Debugger (GDB) is an open source debugger most often used in linux and embedded development. GDB has a command line interface and can only be used from a terminal so in recent times it is usually replaced by a more modern debugger with a graphical user interface. These debuggers are provided by the chip manufacturer most of the time. However as Rust support for STM32 microcontrollers is community driven, selecting GDB as a debugger is a logical step.

GDB communicates with OpenOCD, which acts as a hardware interface and facilitates communication between the host machine and the Cortex-M microcontroller. OpenOCD establishes the link between GDB and the target device, allowing GDB to exert control over the microcontroller for debugging purposes. The debugger supports ARM architectures, including Cortex-M. It understands the specific characteristics and features of these architectures, allowing developers to debug Rust applications targeting Cortex-M microcontrollers effectively.

GDB has all the features of a modern debugger, only developers need to be familiar with the proper commands. It supports symbolic debugging, enabling developers to use high-level constructs like variable names and function names during the debugging process. This abstraction makes it easier to understand and troubleshoot code behavior at a higher level of abstraction. GDB allows developers to set breakpoints at specific lines or functions in the code. It also supports step-by-step execution, enabling users to navigate through the code, line by line, to identify and diagnose issues. During debugging sessions, GDB provides the capability to inspect and modify variable values in real-time. This feature is crucial for understanding the state of the program and making runtime adjustments as needed. GDB allows developers to evaluate expressions and execute commands during a debugging session. This functionality is valuable for dynamically assessing variables or executing specific code snippets to gain insights into the program's behavior. And most importantly as Rust is an LLVM (Low Level Virtual Machine) compatible language it can fully utilize all of the features of an LLDB (Low Level Debugger) such as the GNU Debugger.

\subsection{VSCode}

The final component of this debugger toolchain setup is Visual Studio Code. While VSCode in and of itself is just a feature rich text editor, with the proper extensions and settings, it is able to act as a full fledged IDE including building, flashing, and remote debugging projects. In the previous section GDB was introduced as a complete debugger but it is still missing a convenient graphical user interface. VSCode is able to provide this interface and handle the GDB commands that need to be executed for the provided utilities.

To configure VSCode correctly, the project must contain at least two additional files placed in the root of the project into a \mycode{.vscode} folder and the Cortex-Debug extension. \cite{CortexDebug} The first file is \mycode{tasks.json}. This file can contains multiple tasks that can be executed by commands in Visual Studio Code. In a project like this, normally two tasks are needed. One to build the project using \mycode{cargo} commands and another one that converts the resulting ELF file into a format that can be loaded onto the microcontroller using an interface supported by OpenOCD.
