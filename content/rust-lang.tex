\chapter{The Rust Programming Language}

This chapter is a brief introduction of the Rust programming language, and its most important features in context of this thesis.

\section{A compiled system programming language}

Officially Rust is considered a system programming language which means that its primary function is performance and ease of access to the hardware while still providing higher level programming concepts such as data structures to hold and organize data. \cite{SystemProgrammingLanguageWikipedia} For example, in Rust we can find a dynamic string type, but may also choose to take it as a byte array and manipulate the data that way.

Much like C and C++, Rust programs can only be run after they have been compiled into a binary. The aforementioned higher level concepts only exist at compile time so the binary program can be as efficient as possible. This will be true for all further concepts explained in this section, the resulting binaries from a C and Rust program are comparable both in size and execution speed, its just that the Rust compiler is more sophisticated than the C compiler.

\section{Type system}

\subsection{Types on the stack}

\subsection{Types on the heap}

\subsection{Enums}

\subsection{Type conversions}




\section{Memory safety}

\subsection{Ownership}

\subsection{References (borrowing)}

\subsection{Lifetimes}




\section{Error handling}

\subsection{\mycode{Result} type}

\subsection{\mycode{panic!} macro}
