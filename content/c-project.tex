\chapter{Multicore C Projects}

To create a usable multicore Rust project, a good starting point is studying C language projects created by the official tools of the manufacturer.

\section{Project structure}

Creating C language projects for the STM32H745 microcontroller can be done in the official IDE of STMicroelectronics which is called STMCubeIDE. The software is Eclipse based and can be used to generate empty or example projects, program, and debug the code on the hardware. This is also true for multicore projects, as there are now example projects that use both cores of a H7 series microcontroller. \cite{CExamples}

Creating a dual-core project in STMCubeIDE is similar to how one would create a normal project. By selecting a H7 series MCU not one but three connected projects are generated. A project similar to a single core one will be created for both the M7 and the M4 core. These projects can almost function as standalone projects, they can have their own source files, header files and references. These projects also have a separate \mycode{main.c} file which contains the entry point of the current core. They also both contain builder files, linker files and output folders, so essentially they can be built separately. By default, programming and debugging is also done with these separate projects, so when the IDE is used the cores are programmed one-by-one. Also as a limitation of the software only one of the two cores can be effectively debugged at a time, while the other one either runs or is in a HALT state.

The project is a wrapper for the previous two that rests above them in a hierarchical sense. It can also contain common code that can be referenced and shared from the core-specific projects. Most of the drivers can reside in this project, as regarding peripherals, the M7 and M4 cores are similar. The wrapper project also contains parts of the code that directly refer to dual-core functionalities, for example definitions of some memory locations and even some code regarding dual-core boot can be found there.

In STM32 projects, lots of code around setting up the peripherals is generated during project creation. This is no different in a dual-core project. Setting the function of each pin of the MCU can be done with a GUI, which then generated the code that will configure those pins correctly. The peripherals can be further configured in this GUI by enabling them, an initialization code will be added to the \mycode{main.c} source file of  one of the core-specific projects. In dual-core projects, the core that initializes a peripheral can be selected separately for each one, while also allowing both cores to use the same peripherals when one is enabled for both of them.

Besides the pinout configuration, the clock configuration can also be set in this GUI. This is especially helpful if our project requires different clocks for different clock domains. The clock settings can be configured in a hierarchical graphical interface where the sources and multiplications are visualized on a flow-control diagram. Clock signals for certain domains can even be turned off to save power, if none of the corresponding peripherals are used.
