\pagenumbering{roman}
\setcounter{page}{1}

\selecthungarian

%----------------------------------------------------------------------------
% Abstract in Hungarian
%----------------------------------------------------------------------------
\chapter*{Kivonat}\addcontentsline{toc}{chapter}{Kivonat}

A jelenleg a legelterjedtebb rendszer programozási nyelvek (C/C++) elavultnak tekinthetők, mind biztonságos memóriakezelés, mind modern nyelvi elemek tekintetében. A Rust programozási nyelv megoldást kínál ezen nyelvek sok gyakori problémájára, úgy mint a memóriaszivárgások elkerülésére vagy a nem egységes hibakezelési lehetőségekre. A fejlesztők körében is egyre elfogadottabb a nyelv, hiszen például még a Linux Kernelben is támogatást kapott, a C nyelv után elsőként.\cite{FirstRustCommit}


\vfill
\selectenglish


%----------------------------------------------------------------------------
% Abstract in English
%----------------------------------------------------------------------------
\chapter*{Abstract}\addcontentsline{toc}{chapter}{Abstract}

This document is a \LaTeX-based skeleton for BSc/MSc~theses of students at the Electrical Engineering and Informatics Faculty, Budapest University of Technology and Economics. The usage of this skeleton is optional. It has been tested with the \emph{TeXLive} \TeX~implementation, and it requires the PDF-\LaTeX~compiler.


\vfill
\selectthesislanguage

\newcounter{romanPage}
\setcounter{romanPage}{\value{page}}
\stepcounter{romanPage}